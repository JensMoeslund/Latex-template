\chapter{Introduction}\label{ch:introduction}
The number of \ac{iot} devices is increasing, and expected to exceed 29 billions devices by 2030~\cite{IoTDevices}.

This enables a wide range of new applications, such as smart homes, smart cities, and smart factories.
One of the key technologies in \ac{iot} for enabling these applications is positioning and ranging.
It has been found that \ac{uwb} systems provide better performance in ranging and positioning compared with narrow band systems~\cite{Lu2021}.

This however comes at the cost of more complex hardware, and higher price.
To truly enable the IoT revolution, it is therefore interesting to investigate possibilities for improving the performance of narrowband systems.

There are several different methods such as \ac{tof}, \ac{tdoa}, phase based ranging and received signal strength.
\ac{tof} and \ac{tdoa} have proven effective in the \ac{uwb} system, where the bandwidth
and hence, time resolution is high~\cite{Shen2010}.


In narrowband systems, phase based ranging has been shown to be a very popular and effective method~\cite{Kluge2013,Zand2019}.


Conventional algorithms for phase based ranging perform poorly in environments with many reflecting surfaces, such as indoor environments.
The recent advances in machine learning have made it interesting to investigate if machine learning can be used to improve the performance of phase based ranging in such environments.

This project is proposed by and made in collaboration with Samsung.
The objective of this project is therefore to investigate the possibility of using machine learning to improve the performance of phase based ranging in indoor environments.